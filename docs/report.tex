\documentclass[twocolumn]{article}

\usepackage{booktabs}	% For formal tables
\usepackage{listings}
\usepackage{amsmath}	% AMS Math Package
\usepackage{amsthm} 	% Theorem Formatting
\usepackage{amssymb}	% Math symbols such as \mathbb
\usepackage{siunitx}
\usepackage{graphicx}
\usepackage{siunitx} % deg
\usepackage[margin=2cm]{geometry}

\let\Bbbk\relax


\begin{document}

\title{Implementation of a Freeform Modelling Tool }


\author{Johannes Hauffe, Robert Jendersie}

\maketitle

\section{Introduction}
This report aims to explain implementation details of our multi-resolution freeform modelling tool.
Both in terms of control metaphor and mathematical foundation, we follow the approach described in  \cite{botsch2004intuitive}. Thus, we focus on aspects which are either different in our implementation or only mentioned briefly in the original paper.\\
In addition, we give technical background for the occurring \textit{parameters}, many of which are exposed to the user to allow for experimentation and handling of vastly different meshes.
\section{Smooth Deformation}
Same as in \cite{botsch2004intuitive}, we perform the smooth deformation by minimizing a certain energy functional of the surface with respect to some boundary conditions. For the practical implementation on a triangle mesh we use discretization of the Laplace-Beltrami operator from \cite{meyer2003discrete}. For the vertices $P = \begin{bmatrix}p_1,\dots, p_n\end{bmatrix}^T$ of the mesh, it can be written in matrix form
\begin{equation}\label{laplace}
\triangle = M^{-1} L,
\end{equation}
where $M$ is the diagonal matrix of vertex areas 
\begin{equation*}
m_{ii} = 2 A_{mixed}(p_i),
\end{equation*} and $L$ is the sparse symmetric operator of edge weights $e_{ij}$
\begin{equation*}
l_{ij} = l{ji} = \begin{cases} e_{ij} & i \neq j, \text{edge i-j exists} \\ -\sum_{p_k \in N_1(p_i)} e_{ik} & i=j \\ 0 & \text{else}  \end{cases},
\end{equation*}
where $N_1(p)$ is the one-ring of $p$. To improve numerical robustness for poor triangulations, the edge weights are computed by clamping extreme $\cot$ angles to the range $[0+c,\ang{180}-c]$
\begin{align*}
e_{ij} &= \max(0, f(\cot \alpha_{ij}) + f(\cot \beta_{ij})) \\
f(x) &= \max(\cot c, \min(x, \cot \ang{180}-c)).
\end{align*}
We use $c = \ang{3}$.\\
With \eqref{laplace} we can introduce the higher \textit{order} operators $\triangle^k$ where $k=1$ deforms the surface like a membrane by minimizing the area, $k=2$ characterizes thin plate surfaces which minimize
surface bending and $k=3$ minimizes curvature variation. At the boundary, this behaviour can be smoothly interpolated point wise by introducing diagonal matrices $D_l$ 
\begin{equation}\label{laplace2}
\hat{\triangle}^2 = M^{-1} L D_1 M^{-1} L
\end{equation} ,
\begin{equation}\label{laplace3}
\hat{\triangle}^3 = M^{-1}LD_2 M^{-1} L D_1 M^{-1} L,
\end{equation}
with $d_{k_{ii}} = \lambda_k(p_i)$ from \cite{botsch2004intuitive}.
This value can be adjusted by the user as \textit{smoothness} for boundary and handle points.
Finally, we can look at solving $\hat{\triangle^k} = 0$. Taking only the rows $\hat{\triangle}^k_{sup} = \begin{bmatrix}L_1 L_2\end{bmatrix}$ acting on the support vertices $s$ and the fixed boundary vertices $b$ we get
\begin{equation*}
\begin{bmatrix}L_1 & L_2 \\ 0 & I\end{bmatrix} \begin{bmatrix}s \\ b\end{bmatrix} = \begin{bmatrix} 0 \\ b \end{bmatrix},
\end{equation*}
which leads to the sparse system with non-trivial solution
\begin{equation}\label{lsg}
L_1 s = -L_2 b.
\end{equation}
In all cases $L_1$ is positive definite and in the context of solving the system \eqref{lsg}, both \eqref{laplace} and \eqref{laplace2} can be easily made symmetric, since multiplication by $M$ from the left effectively removes the leftmost $M^{-1}$. For \eqref{laplace3} this only works if $D_1 = D_2$, that is, when no interpolation is done. Thus, if applicable, we employ a sparse $LDL^T$ decomposition and fall back to a sparse $LU$ decomposition for the latter case. \\
Although either decomposition needs to be done just once and solving the system afterwards is relatively fast, introducing precomputed basis functions as described in \cite{botsch2004intuitive} further improves performance.
Instead of picking affinely independent points from the handle $h$, we always use the orthogonal frame
\[
Q\begin{bmatrix}0 & 0 & 0 & 1 \\ 1 & 0 & 0 & 1 \\ 0 & 1 & 0 & 1 \\ 0 & 0 & 1 & 1\end{bmatrix} = \begin{bmatrix} h & 1 \end{bmatrix},
\]
to find the matrix $Q \in \mathbb{R}^{H \times 4}$ of affine combinations.
\section{Detail Preservation}
We implement a multi-resolution editing approach based on point wise displacement vectors, as described in \cite{kobbelt1998interactive}.
To preserve high frequency components in the support area, the details need to be first extracted and then reapplied to the modified mesh. We either use the displacement resulting from the initial solution of \eqref{lsg} without changes to the handle, or perform \textit{implicit smoothing} of the form
\begin{equation}
(I - dt \triangle^k) s = s_0,
\end{equation}
to the support region with initial points $s_0$. The choice of a reasonable time-step $dt$ varies per mesh and can be adjusted by the user as \textit{strength}.
To maintain the full details, the \textit{smoothing order} $k$ should be the same as for the deformation, but in some cases a different order is more robust.
Similar to \eqref{lsg}, we can integrate the fixed boundary and make the system symmetric to then solve
\begin{equation}
(M_1 - dt L_1) s = M_1 s_0 + dt L_2 b,
\end{equation}
with the sparse $LDL^T$ decomposition, where $M_1$ are the area weights of vertices associated with $s$.
Since the Laplace-Beltrami operator \eqref{laplace} is used, the points should only move in normal direction.
Thus, the resulting displacement is encoded per vertex in a local frame, defined by its normal and one edge. Optionally, other vertices in the \textit{n-ring} can be considered to find the local frame where the displacement vector has the smallest length. Also proposed in \cite{kobbelt1998interactive}, this leads to fewer anomalies in the reconstructed surface, as seen in Figure~\ref{fig:searchRing}.
\begin{figure}
	\includegraphics[width=0.5\textwidth]{searchframe.png}
	\caption{Detail reconstruction using the local frame of each vertex (left) and the frame with the shortest displacement vector in the 4-ring neighbourhood (right). }
	\label{fig:searchRing}
\end{figure}

\section{Results}

For a deformation to be performed in real time, the algorithm must have a good performance.
To verify the performance of the algorithm different scenarios were tested with regards to changing parameters and doing position changes (translation, rotation or scaling).
All scenarios were tested on a mesh with 125k vertices, where the handle region contains 15k and the support region 35k vertices. The measured times are the average of 500 measured values. \\
If we consider only the vertex position changes, then these can be done in real time. The only difference is if we encode with or without details.

\begin{table}[h]
	\centering
	\caption{Hardware of the test system}
	\begin{tabular}{|l|l|}
		\hline
		System & Windows 10 Pro 64bit \\
		CPU & i5-6600k, 4 Cores, 3.50 GHz \\
		RAM & 16 GB DDR4 \\
		\hline
	\end{tabular}
\label{tab:handle}
\end{table}



\begin{table}[h]
	\centering
	\caption{Performance of position changes of vertices.}
	\begin{tabular}{|l|l|}
		\hline
		without details & 1.632 ms \\
		with details & 23.232 ms \\
		\hline
	\end{tabular}
\label{tab:handle}
\end{table}

Apart from the position changes, you can also make changes that change the deformation operator. These changes can only be used with a delay.

\begin{table}[h]
	\centering
	\caption{The results of the measurements when we change the order of the operator}
	\begin{tabular}{|l|l|}
		\hline
		order 1 & 276.769 ms \\
		order 2 & 922.871 ms \\
		order 3 & 2262.19 ms \\
		\hline
		details order 1 & 212.810 ms \\
		details order 2 & 830.765 ms \\
		details order 3 & 2168.810 ms \\
		\hline
	\end{tabular}
\label{tab:order}
\end{table}

It is possible to adjust the smoothness factor to change the behaviour at the border regions at the handle an support.
The changes are also only achievable with a noticeable delay.

\begin{table}[h]
	\centering
	\caption{Results when we change the configuration of the behaviour at the border regions}
	\begin{tabular}{|l|l|}
		\hline
		detail strength & 161.554 ms \\
		\hline
		search ring 1 & 89.594 ms \\
		search ring 5 & 239.423 ms \\
		search ring 10 & 725.021 ms \\
		search ring 20 & 2417.368 ms \\
		\hline
		smoothness handle & 271.178 ms \\
		smoothness boundary & 268.972 ms \\		
		\hline
	\end{tabular}
\label{tab:order}
\end{table}

\bibliographystyle{ieeetr}
\bibliography{references}

\end{document}
